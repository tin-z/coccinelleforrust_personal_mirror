\documentclass[aspectratio=169]{beamer}
\usepackage[all]{xy}
\usetheme{metropolis}

\usepackage{graphicx}
\usepackage{bm}
\usepackage{amsmath}
\usepackage{beamerfoils}
\usepackage{hyperref}
\usepackage{listings}
\usepackage{textpos}
\usepackage{alltt}
\usepackage{color}
\usepackage{tikz}
\usepackage{pgfplots}
\pgfplotsset{compat=newest,
   /pgfplots/ybar legend/.style={
    /pgfplots/legend image code/.code={%
       \draw[##1,/tikz/.cd,yshift=-0.25em]
        (0cm,0cm) rectangle (2pt,0.8em);},
   },
}

\definecolor{color1}{rgb}{0.549,0.815,0.960}
\definecolor{color2}{rgb}{0.607,0.937,0.498}
\definecolor{color3}{rgb}{0.839,0.929,0.427}
\definecolor{color4}{rgb}{0.878,0.619,0.141}
\definecolor{color5}{rgb}{0.396,0.165,0.055}
\definecolor{color6}{rgb}{0.643,0.368,0.898}
\definecolor{colortop}{rgb}{0.878,0.141,0.141}

\usepackage{wasysym}

\newcommand{\snest}{Nest}

\newcommand{\extrabold}{}%{\bfseries}

  \lstdefinelanguage{diff}{
	basicstyle=\ttfamily\extrabold\scriptsize,
	morecomment=[f][\color{subtitlex}]{@},
	morecomment=[f][\color{gr}]{+},
	morecomment=[f][\color{red}]{-},
	morecomment=[f][\color{purple}]{*},
	morecomment=[f][\color{purple}]{?},
        keepspaces=true,
	escapechar=£,
	identifierstyle=\color{black},
  }

  \lstdefinelanguage{tdiff}{
	basicstyle=\ttfamily\extrabold\tiny,
	morecomment=[f][\color{subtitlex}]{@},
	morecomment=[f][\color{gr}]{+},
	morecomment=[f][\color{red}]{-},
        keepspaces=true,
	escapechar=£,
	identifierstyle=\color{black},
  }

\lstloadlanguages{C}
\lstset{language=C,
	basicstyle=\ttfamily\extrabold\scriptsize,
	backgroundcolor=\color{white},
        showspaces=false,
        rulesepcolor=\color{gray},
	showstringspaces=false,
	keywordstyle=\bfseries\color{blue!40!black},
	commentstyle=\itshape\color{purple!40!black},
	identifierstyle=\color{black},
	stringstyle=\color{red},
        numbers=none,
        numbersep=2pt,
        morekeywords={elif},
       }

%\lstset{escapeinside={(*@}{@*)},style=customP}

\newcommand{\ttlb}{\mbox{\tt \char'173}}
\newcommand{\ttrb}{\mbox{\tt \char'175}}
\newcommand{\ttbs}{\mbox{\tt \char'134}}
\newcommand{\ttmid}{\mbox{\tt \char'174}}
\newcommand{\tttld}{\mbox{\tt \char'176}}
\newcommand{\ttcar}{\mbox{\tt \char'136}}
\newcommand{\msf}[1]{\mbox{\sf{{#1}}}}
\newcommand{\mita}[1]{\mbox{\it{{#1}}}}
\newcommand{\mbo}[1]{\mbox{\bf{{#1}}}}
\newcommand{\mth}[1]{\({#1}\)}
\newcommand{\ssf}[1]{\mbox{\scriptsize\sf{{#1}}}}
\newcommand{\sita}[1]{\mbox{\scriptsize\it{{#1}}}}
\newcommand{\mrm}[1]{\mbox{\rm{{#1}}}}
\newcommand{\mtt}[1]{\mbox{\tt{{#1}}}}

\definecolor{gr}{rgb}{0.22,0.63,0.08}
\definecolor{subtitlex}{rgb}{0.2,0.2,0.7}

\setlength{\fboxrule}{1.75pt}
\newcommand{\mybox}[2]{\fcolorbox{#1}{black!2}{\makebox[1.2\height]{\begin{tabular}{c}\textcolor{#1}{#2}\end{tabular}}}}
\newcommand{\emptybox}[1]{\mybox{#1}{\textcolor{black!2}{t1}}}
\newcommand{\blackbox}{\emptybox{black}}
\newcommand{\grbox}[1]{\mybox{gr}{T{#1}}}
\newcommand{\redbox}[1]{\mybox{subtitlex}{\textcolor{red}{T{#1}}}}
\newcommand{\gredbox}[1]{\mybox{gr}{\textcolor{red}{T{#1}}}}
\newcommand{\bluebox}[1]{\mybox{subtitlex}{T{#1}}}

%\\[-2mm]
%5\\[-2mm]{}

\title{Coccinelle for Rust\\
https://gitlab.inria.fr/coccinelle/coccinelleforrust.git
}
\author{Julia Lawall, Tathagata Roy}
\date{November 15, 2023\\ \mbox{}
}

\begin{document}

\frame{\titlepage}
\frame{\frametitle{Goals}

\begin{itemize}
\setlength{\itemsep}{4mm}
\item Perform repetitive transformations at a large scale. \pause
\begin{itemize}
\setlength{\itemsep}{2mm}
\item[--] Rust is 1.6 MLOC.
\item[--] The Linux kernel is 23 MLOC. \pause
\item[--] Collateral evolutions: a change in an API requires changes in all clients.
\end{itemize}

\pause

\item Provide a transformation language that builds on developer expertise.

\pause

\item Changes + developer familiarity = (semantic) patches
\end{itemize}
}

\frame[containsverbatim]{\frametitle{An example change (Rust
    repository)}

\begin{lstlisting}[language=tdiff]
commit d822b97a27e50f5a091d2918f6ff0ffd2d2827f5
Author: Kyle Matsuda <kyle.yoshio.matsuda@gmail.com>
Date:   Mon Feb 6 17:48:12 2023 -0700

    change usages of type_of to bound_type_of

diff --git a/compiler/rustc_borrowck/src/diagnostics/conflict_errors.rs b/compiler/.../conflict_errors.rs
@@ -2592,4 +2592,4 @@ fn annotate_argument_and_return_for_borrow(
             } else {
-                £\textcolor{red}{let ty = }£self.infcx.tcx.type_of(self.mir_def_id())£\textcolor{red}{;}£
+                £\textcolor{gr}{let ty = }£self.infcx.tcx.bound_type_of(self.mir_def_id()).subst_identity()£\textcolor{gr}{;}£
                 match ty.kind() {
                     ty::FnDef(_, _) | ty::FnPtr(_) => self.annotate_fn_sig(
diff --git a/compiler/rustc_borrowck/src/diagnostics/mod.rs b/compiler/.../mod.rs
@@ -1185,4 +1185,4 @@ fn explain_captures(
                         matches!(tcx.def_kind(parent_did), rustc_hir::def::DefKind::Impl { .. })
                             .then_some(parent_did)
-                            £\textcolor{red}{.and\_then(|did| match}£ tcx.type_of(did)£\textcolor{red}{.kind() \ttlb}£
+                            £\textcolor{gr}{.and\_then(|did| match}£ tcx.bound_type_of(did).subst_identity()£\textcolor{gr}{.kind() \ttlb}£
                                 ty::Adt(def, ..) => Some(def.did()),
...
\end{lstlisting}

\textcolor{subtitlex}{136 files changed, 385 insertions(+), 262 deletions(-)}
}

\frame[containsverbatim]{\frametitle{An example change (Rust
    repository)}

\begin{lstlisting}[language=tdiff]
commit d822b97a27e50f5a091d2918f6ff0ffd2d2827f5
Author: Kyle Matsuda <kyle.yoshio.matsuda@gmail.com>
Date:   Mon Feb 6 17:48:12 2023 -0700

    change usages of type_of to bound_type_of

diff --git a/compiler/rustc_borrowck/src/diagnostics/conflict_errors.rs b/compiler/.../conflict_errors.rs
@@ -2592,4 +2592,4 @@ fn annotate_argument_and_return_for_borrow(
             } else {
-                £\textcolor{black}{let ty = }£self.infcx.tcx.type_of(self.mir_def_id())£\textcolor{black}{;}£
+                £\textcolor{black}{let ty = }£self.infcx.tcx.bound_type_of(self.mir_def_id()).subst_identity()£\textcolor{black}{;}£
                 match ty.kind() {
                     ty::FnDef(_, _) | ty::FnPtr(_) => self.annotate_fn_sig(
diff --git a/compiler/rustc_borrowck/src/diagnostics/mod.rs b/compiler/.../mod.rs
@@ -1185,4 +1185,4 @@ fn explain_captures(
                         matches!(tcx.def_kind(parent_did), rustc_hir::def::DefKind::Impl { .. })
                             .then_some(parent_did)
-                            £\textcolor{black}{.and\_then(|did| match}£ tcx.type_of(did)£\textcolor{black}{.kind() \ttlb}£
+                            £\textcolor{black}{.and\_then(|did| match}£ tcx.bound_type_of(did).subst_identity()£\textcolor{black}{.kind() \ttlb}£
                                 ty::Adt(def, ..) => Some(def.did()),
...
\end{lstlisting}

\textcolor{subtitlex}{136 files changed, 385 insertions(+), 262 deletions(-)}
}

\frame[containsverbatim]{\frametitle{Creating a semantic patch: Step 1: remove irrelevant code}

\begin{lstlisting}[language=tdiff]
££
££
££
££
££
££
££
££
££
-                £\textcolor{black!2}{let ty = }£self.infcx.tcx.type_of(self.mir_def_id())£\textcolor{black!2}{;}£
+                £\textcolor{black!2}{let ty = }£self.infcx.tcx.bound_type_of(self.mir_def_id()).subst_identity()£\textcolor{black!2}{;}£
££
££
££
££
££
££
-                            £\textcolor{black!2}{.and\_then(|did| match}£ tcx.type_of(did)£\textcolor{black!2}{.kind() \ttlb}£
+                            £\textcolor{black!2}{.and\_then(|did| match}£ tcx.bound_type_of(did).subst_identity()£\textcolor{black!2}{.kind() \ttlb}£
££
££
\end{lstlisting}

\textcolor{black!2}{136 files changed, 385 insertions(+), 262 deletions(-)}
}

\frame[containsverbatim]{\frametitle{Creating a semantic patch: Step 2:
    pick a typical example}

\begin{lstlisting}[language=tdiff]
@@

@@

- self.infcx.tcx.type_of(self.mir_def_id())
+ self.infcx.tcx.bound_type_of(self.mir_def_id()).subst_identity()
\end{lstlisting}

\vfill

\textcolor{black!2}{Updates over 200 call sites.}
}

\frame[containsverbatim]{\frametitle{Creating a semantic patch: Step 3: abstract over subterms using metavariables}

\begin{lstlisting}[language=tdiff]
@@
expression tcx, arg;
@@

- tcx.type_of(arg)
+ tcx.bound_type_of(arg).subst_identity()
\end{lstlisting}

\vfill

\textcolor{black!2}{Updates over 200 call sites.}
}

\addtocounter{framenumber}{-1}

\frame[containsverbatim]{\frametitle{Creating a semantic patch: Step 3: abstract over subterms using metavariables}

\begin{lstlisting}[language=tdiff]
@@
expression tcx, arg;
@@

- tcx.type_of(arg)
+ tcx.bound_type_of(arg).subst_identity()
\end{lstlisting}

\vfill

\textcolor{subtitlex}{Updates over 200 call sites.}
}

\frame[t,containsverbatim]{\frametitle{An outlier}

\begin{lstlisting}[language=tdiff]
             let (shim_size, shim_align, _kind) = ecx.get_alloc_info(alloc_id);
+            let def_ty = ecx.tcx.bound_type_of(def_id).subst_identity();
             let extern_decl_layout =
-                ecx.tcx.layout_of(ty::ParamEnv::empty().and(ecx.tcx.type_of(def_id))).unwrap();
+                ecx.tcx.layout_of(ty::ParamEnv::empty().and(def_ty)).unwrap();
             if extern_decl_layout.size != shim_size || extern_decl_layout.align.abi != shim_align {
                 throw_unsup_format!(
                     "`extern` static `{name}` from crate `{krate}` has been declared \
\end{lstlisting}

\vspace{0.5\baselineskip}

\textcolor{black!2}{The developer has created a new name to avoid a long
  line.}
\begin{itemize}
\item[] \textcolor{black!2}{Could address it manually.}
\item[] \textcolor{black!2}{Could create a rule for the special case of nested function call
  contexts \newline (probably not worth it for one case).}
\end{itemize}
}

\addtocounter{framenumber}{-1}

\frame[t,containsverbatim]{\frametitle{An outlier}

\begin{lstlisting}[language=tdiff]
             let (shim_size, shim_align, _kind) = ecx.get_alloc_info(alloc_id);
+            let def_ty = ecx.tcx.bound_type_of(def_id).subst_identity();
             let extern_decl_layout =
-                ecx.tcx.layout_of(ty::ParamEnv::empty().and(ecx.tcx.type_of(def_id))).unwrap();
+                ecx.tcx.layout_of(ty::ParamEnv::empty().and(def_ty)).unwrap();
             if extern_decl_layout.size != shim_size || extern_decl_layout.align.abi != shim_align {
                 throw_unsup_format!(
                     "`extern` static `{name}` from crate `{krate}` has been declared \
\end{lstlisting}

\vspace{0.5\baselineskip}

\textcolor{subtitlex}{The developer has created a new name to avoid a long
  line.}
\begin{itemize}
\item Could address it manually.
\item Could create a rule for the special case of nested function call
  contexts \newline (probably not worth it for one case).
\end{itemize}
}

\frame[t,containsverbatim]{\frametitle{An alternate semantic patch}

\begin{lstlisting}[language=tdiff]
@@
expression tcx, arg;
@@

££
  tcx.
-    type_of(arg)
+    bound_type_of(arg).subst_identity()
\end{lstlisting}

\vfill
\textcolor{subtitlex}{Putting tcx in the context ensures any comments will
  be preserved.}
}


\frame[t,containsverbatim]{\frametitle{A refinement}

\begin{lstlisting}[language=tdiff]
@@
TyCtxt tcx;
expression arg;
@@

  tcx.
-    type_of(arg)
+    bound_type_of(arg).subst_identity()
\end{lstlisting}

\vfill

\textcolor{subtitlex}{Specifying the type of {\tt tcx} protects against
  changing other uses of \texttt{type\_of}.} 
}

\frame{\frametitle{Some Coccinelle internals}

\textcolor{subtitlex}{Input:} Parsing provided by Rust Analyzer.
\begin{itemize}
\setlength{\itemsep}{2mm}
\item Used both for Rust code and for semantic patch code.
\item Will provide type inference, when needed (currently, loses concurrency).
\end{itemize} \pause

\vspace{\baselineskip}

\textcolor{subtitlex}{Output:} Pretty printing provided by {\tt rustfmt}.
\begin{itemize}
\setlength{\itemsep}{2mm}
\item To avoid problems with code not originally formatted with {\tt
  rustfmt} \newline (or formatted with a different version), the {\tt rustfmt}ed
  changes are dropped back into the original code.
\item Preserves comments and whitespace in the unchanged part of the code.
\end{itemize}}

\frame{\frametitle{Some Coccinelle internals}
\textcolor{subtitlex}{In the middle:}
\begin{itemize}
\setlength{\itemsep}{3mm}
\item Wrap Rust code and semantic patch code, eg to indicate metavariables.
\item Match semantic patch code against Rust code, to collect change sites and metavariable bindings.
\item On a successful match, apply the changes, instantiated according to the metavariable bindings, reparse, and repeat with the next rule.
\end{itemize}}

\frame{\frametitle{A case study}

\textcolor{subtitlex}{Software:} stratisd
\begin{itemize}
\item https://github.com/stratis-storage/stratisd
\item Easy to use local storage management for Linux.
\item Over 2000 commits, and over 10K lines of Rust code.
\end{itemize}

\vspace{\baselineskip}

\textcolor{subtitlex}{Commit selection:}
\begin{itemize}
\item Patchparse: https://gitlab.inria.fr/lawall/patchparse4
\item Collect change patterns that occur at least 40 times.
\item 13 commits selected, affecting 10-94 files, and up to 3000 $\textcolor{gr}{+}/\textcolor{red}{-}$ lines.
\end{itemize}}

\frame[containsverbatim]{\frametitle{Some successes}

\textcolor{subtitlex}{Commits:}
\begin{itemize}
\item 39b925b0: Remove EngineError alias
\item c3918972: Replace EngineResult usage with StratisResult
\end{itemize}

\textcolor{subtitlex}{Semantic patch:}
\begin{columns}
\begin{column}{0.3\textwidth}
\begin{lstlisting}[language=diff,backgroundcolor=\color{black!2}]
@type@
@@
- EngineError
+ StratisError
\end{lstlisting}
\end{column}
\begin{column}{0.3\textwidth}
\begin{lstlisting}[language=diff,backgroundcolor=\color{black!2}]
@type@
@@
- EngineResult
+ StratisResult
\end{lstlisting}
\end{column}
\end{columns}

\vspace{0.5\baselineskip}

\textcolor{black!2}{Results:}
\begin{itemize}
\item[] \textcolor{black!2}{{Typical changes:} {\tt use}, method signatures, method calls.}
\item[] \textcolor{black!2}{Benefits from recent improvements in pretty printing.}
\end{itemize}}

\frame[containsverbatim]{\frametitle{Some successes}

\textcolor{subtitlex}{Commits:}
\begin{itemize}
\item 39b925b0: Remove EngineError alias
\item c3918972: Replace EngineResult usage with StratisResult
\end{itemize}

\textcolor{subtitlex}{Semantic patch:}
\begin{columns}
\begin{column}{0.3\textwidth}
\begin{lstlisting}[language=diff,backgroundcolor=\color{black!2}]
@type@
@@
- EngineError
+ StratisError
\end{lstlisting}
\end{column}
\begin{column}{0.3\textwidth}
\begin{lstlisting}[language=diff,backgroundcolor=\color{black!2}]
@type@
@@
- EngineResult
+ StratisResult
\end{lstlisting}
\end{column}
\end{columns}

\vspace{0.5\baselineskip}

\textcolor{subtitlex}{Results:}
\begin{itemize}
\item {Typical changes:} {\tt use}, method signatures, method calls.
\item Benefits from recent improvements in pretty printing.
\end{itemize}}

\frame[containsverbatim]{\frametitle{Some successes}

\textcolor{subtitlex}{fe7df6a9:} Remove unnecessary pub modifier on
stratisd tests

\vspace{0.4\baselineskip}

\textcolor{subtitlex}{Semantic patch:}
\begin{lstlisting}[language=diff,backgroundcolor=\color{black!2}]
@@
identifier f;
expression e;
@@
#[test]
- pub
  fn f() { e; }
\end{lstlisting}

\textcolor{subtitlex}{Results:}
\begin{itemize}
\item 69 changes across 9 files.
\item 1 case has an additional attribute and thus is omitted.
\end{itemize}}

\frame[containsverbatim]{\frametitle{Some successes}

\textcolor{subtitlex}{9c60ad44:} Remove ErrorEnum and add error chaining

\begin{columns}
\begin{column}{0.5\textwidth}
\begin{lstlisting}[language=tdiff,backgroundcolor=\color{black!2}]
@@
expression return_message, e1;
@@
  return_message.append3(e1,
-   msg_code_ok(), msg_string_ok(),
+   DbusErrorEnum::OK as u16, OK_STRING.to_string(),
  )

@@
@@
- DbusErrorEnum::INTERNAL_ERROR
+ DbusErrorEnum::ERROR
\end{lstlisting}
\end{column}
\begin{column}{0.3\textwidth}
\begin{lstlisting}[language=tdiff,backgroundcolor=\color{black!2}]
@@
expression e;
@@
- StratisError::Error
+ StratisError::Msg
    (e,)

@@
expression e1, e2;
@@
- StratisError::Engine(e1,
+ StratisError::Msg(
    e2,)
\end{lstlisting}
\end{column}
\end{columns}

\textcolor{subtitlex}{Results:}
\begin{itemize}
\item Covers 290/417 changes. Omits {\tt use}s and some less common patterns.
\item Trailing commas lead to a lot of rule duplication.
\end{itemize}}

\frame[containsverbatim]{\frametitle{Some successes}

\textcolor{subtitlex}{d4ac5d89:} Switch from trait objects to type parameters and associated types

%\textcolor{subtitlex}{Semantic patch extract:}
\begin{lstlisting}[language=tdiff,backgroundcolor=\color{black!2}]
@r1@
identifier mthd, f;
identifier t;
@@
pub fn
- mthd(f: &Factory<MTSync<TData>, TData>,) -> t<MTSync<TData>, TData>
+ mthd<E>(f: &Factory<MTSync<TData<E>>, TData<E>>,) -> t<MTSync<TData<E>>, TData
<E>>
+where E: 'static + Engine,
  {
  ...
  }
\end{lstlisting}

\textcolor{subtitlex}{Results:}
\begin{itemize}
\item Covers ??? changes.
\item Trailing commas issues.
\item New feature: \texttt{\ldots} for method bodies.
\begin{itemize}
\item[--] Matches both simple expressions and block expressions.
\end{itemize}
\end{itemize}}

\frame{\frametitle{Some failures}

\textcolor{subtitlex}{Commits:}
\begin{itemize}
\item aeed4b7c: Use inline format arguments
\item ea33caf4: Conform to snake\_case naming style
\end{itemize}

\vspace{\baselineskip}\pause

\textcolor{subtitlex}{Issues:}
\begin{itemize}
\item Require changes inside identifier names and strings.
\item Such changes require scripting, as found in Coccinelle for C.
\end{itemize}
}

\frame[containsverbatim]{\frametitle{Some failures}

\textcolor{subtitlex}{2569545c:} Add anonymous lifetime parameters.

\vspace{0.5\baselineskip}

\textcolor{subtitlex}{Semantic patch extract:}
\begin{lstlisting}[language=diff,backgroundcolor=\color{black!2}]
@type@
lifetime l1,l2;
@@
(
App <l1,l2>
|
App
+ <'_,'_>
)
\end{lstlisting}

\vspace{0.5\baselineskip}

\textcolor{subtitlex}{Disjunctions on types not currently supported.}
}

\frame[containsverbatim]{\frametitle{Some failures}

\textcolor{subtitlex}{f00fb860:} Allow disabling actions when stratisd detects unresolvable failures

\vspace{0.5\baselineskip}

\textcolor{subtitlex}{Semantic patch extract:}
\begin{lstlisting}[language=tdiff,backgroundcolor=\color{black!2}]
@@
identifier mthd;
@@
- log_action!(
+ handle_action!(
    pool.mthd(
             ...
             )
+   ,dbus_context, pool_path.get_name()
    )
\end{lstlisting}

\textcolor{subtitlex}{Issues:}
\begin{itemize}
\item This covers a few changes, but the commit has more variety.
\item New feature: {\tt \ldots} for argument lists and parameter lists.
\item Future feature: {\tt \ldots} to connect the definitions of pool\_path
  to the call site.
\end{itemize}}

\frame{\frametitle{Discussion}

\begin{itemize}
\setlength{\itemsep}{5mm}
\item Rust projects of interest?
\item Transformations of interest?
\end{itemize}
}

\frame{\frametitle{Conclusion}

\begin{itemize}
\setlength{\itemsep}{5mm}
\item Pattern-based transformation language.
\begin{itemize}
\item[--] Changes can be expressed in all parts of the code: expressions,
  signatures, lifetimes, etc.
\item[--] Changes can be sensitive to expression types.
\end{itemize}
\item Works well for frequent atomic changes.
\begin{itemize}
\item[--] Recent updates to improve pretty printing, handling of macros,
  genericity ({\tt \ldots}), etc.
\end{itemize}
\item Future work: {\tt \ldots} for control-flow paths, nesting.
\begin{itemize}
\item[--] Connect variable definitions to uses.
\item[--] Connect method definitions to the containing type implementation.
\end{itemize}
\end{itemize}

\begin{center}
\textcolor{red}{\bf https://gitlab.inria.fr/coccinelle/coccinelleforrust.git}
\end{center}
}
\end{document}
